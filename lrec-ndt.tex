\documentclass[11pt,a4paper]{article}

\usepackage{ctable}
\usepackage[T1]{fontenc}
\usepackage[utf8]{inputenc}
\usepackage{lrec2006}
\usepackage{apacite}

\bibliographystyle{apacite}

\title{The Norwegian Dependency Treebank}
\author{Folk}

\begin{document}
\maketitle

\section{Introduction (Lilja)}
A syntactic treebank constitutes an important language resource in
establishing a set of natural language processing tools for a
language and may be employed for central tasks such as part-of-speech
tagging and syntactic parsing. For the past decade, dependency
analysis has become an increasinlgy popular form of syntactic analysis
and has been claimed to strike a balance between a depth of analysis
sufficient for many down-stream applications and accuracy and
efficiency in parsing with these types representations. The CoNLL
shared tasks devoted to dependency parsing and joint syntactic and
semantic parsing,
\cite<see e.g.,>{Niv:Hal:Kub:07,Haj:Cia:Joh:09}, have been instrumental in
establishing a common set of dependency treebanks for a range of
languages such as English, Swedish, Czech and Arabic, \footnote{Note
  that several of the treebanks employed for the shared task were not
  originally dependency treebanks, but rather converted from
  phrase-structure representations.}  thus enabling multilingual evaluation of different systems.  The increased
availability of dependency parsers has spurred down-stream use of
dependency representations in diverse tasks such as Machine Translation \shortcite{Din:Pal:05}, Sentiment Analysis \shortcite{Wil:Wie:Hof:09} and Negation Resolution \shortcite{Lap:Vel:Ovr:12}.

Until recently, there has been no publicly available (dependency)
treebank for Norwegian.\footnote{Possibly comment on INESS?} Hence,
the progress in parsing and applications described above have not been
possible for Norwegian.  At present, however, the Language Bank,
hosted at the Norwegian National Library, is in the process of
completing a two year project with the aim of producing a dependency
treebank for Norwegian. 

In this paper we present the Norwegian Dependency Treebank (NDT), a
syntactic treebank which encompasses treebanks for both variants of
Norwegian (Bokm{\aa}l and Nynorsk)\footnote{On the distinction between
  BM and NN}. In the following, we describe the main annotation
principles employed in the syntactic analysis of the treebank and
discuss the selection of texts. We then go on to describe the
annotation process in some detail. Finally, we present first results
for data-driven dependency parsing of Norwegian.

\section{Annotation principles (PE)}
\begin{itemize}
    \item Leksikalske/funksjonelle hoder
    \item Verbpartikler
    \item Selekterte vs. uselekterte PP
    \item Greatest hits
    \item Eksempelgraf
\end{itemize}

\section{Texts}
\ctable[botcap,
    caption={Statistics for texts in the Bokmål section},
    label=tbl:stats-bm
]{lrr}{}{
        \FL
        Source          & Sentences & Tokens \ML
        Aftenposten     &  5646 &  92313 \NN
        Blogg           &   706 &  22330 \NN
        Bergens Tidende &  2551 &  38900 \NN
        Dagbladet       &  4436 &  60720 \NN
        Klassekampen    &   919 &  16486 \NN
        NOU             &   477 &   8360 \NN
        Sunnmørsposten  &  1201 &  19109 \NN
        Storting        &  1280 &  22146 \NN
        VG              &   868 &  12821 \ML
        Total           & 18084 & 282075
        \LL
    }

\ctable[botcap,
    caption={Statistics for texts in the Nynorsk section},
    label=tbl:stats-nn
]{lrr}{}{
        \FL
        Source             & Sentences & Tokens \ML
        Blogg              &   579 &   9676 \NN
        Dag og Tid         &  5278 &  88982 \NN
        Firda              &  3401 &  50579 \NN
        Klassekampen       &  2215 &  39739 \NN
        Mål og Meining     &   976 &  25113 \NN
        Vest-Telemark blad &  2956 &  46990 \ML
        Total              & 15405 & 263079
        \LL
    }

Commodo voluptate laboris, dolor Tonx whatever trust fund. Salvia cliche plaid
Brooklyn, veniam Terry Richardson cardigan slow-carb craft beer ullamco
tattooed. Wayfarers id banh mi jean shorts ex deserunt. Master cleanse beard
minim whatever. Non farm-to-table organic duis. Truffaut Neutra sriracha,
ethnic adipisicing put a bird on it eiusmod Helvetica keffiyeh American
Apparel cred selvage. Keffiyeh chia banjo Banksy laborum odio twee, craft beer
labore gentrify salvia pour-over jean shorts Echo Park PBR.

\section{Annotation process}
\subsection{Annotators and annotation work flow}
\subsection{Preprocessing}
\subsection{Syntactic preprocessing (Arne)}
\subsection{Inter-annotator agreement (Arne)}

\section{Dependency parsing (Arne \& Lilja)}
Wes Anderson elit butcher disrupt lomo, selvage mlkshk pickled whatever seitan
nisi squid sriracha. Eiusmod vero Austin, incididunt pour-over Odd Future
squid hashtag. Sint twee gastropub, qui velit direct trade lo-fi cornhole
beard wayfarers commodo put a bird on it gluten-free veniam. Umami paleo
kitsch quis, qui semiotics ea vero. Irony lo-fi cliche sunt. Mustache before
they sold out cliche, polaroid Blue Bottle voluptate aliquip swag hashtag
irony adipisicing. 8-bit ennui eu, +1 actually lo-fi aute adipisicing esse.

%\newpage
%\onecolumn
\clearpage
\bibliography{ndt}

\end{document}
