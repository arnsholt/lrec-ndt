\documentclass{ltgposter12}

\usepackage{expex}
\usepackage[T1]{fontenc}
\usepackage[utf8]{inputenc}

%% Alternativ stil på boksene.
%\newskip\boxskip
%\boxskip10em % Ekstra mellomrom mellom boksene.
%
%\setbeamertemplate{block begin}{
%  \vskip\boxskip
%  {\ifbeamercolorempty[bg]{block body}{}{\nointerlineskip\vskip-0.5pt}}%
%  \begin{beamercolorbox}[rounded=false,shadow=false,leftskip=0.5em,sep=0.5em]{block body}%
%    \usebeamerfont{block body}%
%    \begin{center}
%      \usebeamerfont*{block title}\insertblocktitle
%    \end{center}
%    \ifbeamercolorempty[bg]{block body}{\vskip-.25ex}{\vskip-.75ex}\vbox{}%
%  }
%\setbeamertemplate{block end}{
%  \end{beamercolorbox}
%}
%
%\setbeamerfont*{itemize/enumerate subbody}{parent=itemize/enumerate body}
%\setbeamerfont*{itemize/enumerate subsubbody}{parent=itemize/enumerate body}

\title{The Norwegian Dependency Treebank}
\author{\bf Per Erik Solberg$^{\ast}$, Arne Skj{\ae}rholt$^{\dagger}$, 
        Lilja {\O}vrelid$^{\dagger}$, Kristin Hagen$^{\ddag}$, 
        and Janne Bondi Johannessen$^{\ddag}$}
\institute{$^{\ast}$ Spr{\aa}kbanken, The National Library of Norway;
    $^{\dagger}$ Department of Informatics, University of Oslo;
    $^{\ddag}$ Department of Linguistics and Scandinavian Studies, University
        of Oslo}
\conference{LREC 2014}

\begin{document}
\begin{columns}[t]
    \begin{column}{0.25\textwidth}
        \begin{block}{Introduction}
            \begin{itemize}
              \item The Norwegian Dependency Treebank (NDT) is a new syntactic treebank for Norwegian Bokmål and Norwegian Nynorsk, developed at the National Library of Norway
              \item Manually annotated with pos-tags, morphological features, syntactic functions and dependency graphs
              \item 600 000 tokens, equally distributed between Bokmål and Nynorsk
              \item Mostly newspaper text, but also small portions of texts from government reports, parliament transcripts and blogs
              \item Consistency and parsability comparable to other treebank projects
            \end{itemize}
        \end{block}

        \begin{block}{Annotation principles}
          \begin{enumerate}
           \item \textbf{Linguistic adequacy:} The annotation should be as linguistically adequate as possible.
           \item \textbf{Consistency:} It had to be possible for annotators to implement the analyses consistently.
           \item \textbf{Quick annotation:} As size matters for most users, the annotators should be able to annotate quickly.
           \item \textbf{Easy retrieval:} It should be easy to retrieve specific constructions.
          \end{enumerate}
        \end{block}
        \begin{block}{Annotation example 1: Complementizers}
%          \begin{itemize}
%           \item Subordinate clauses are always headed by the finite verb, not the complementizer, cf. (\ref{medat}) and the analysis in figure ?????
            \ex\label{medat}\begingl
                \gla Nå tror lokale myndigheter at bortføringen var nøye planlagt. //
                \glb now believe local authorities that.comp abduction+the was carefully planned //
                \glft`Local authorities now believe that the abduction was carefully planned.' //
           \endgl\xe
%           \item This makes it possible to have a unified analysis of subordinate clauses with and without a complementizer, cf. (\ref{utenat}) and the analysis in figure ????
            \ex\label{utenat}\begingl
                \gla Jeg tror ikke det er tilfeldig. //
                \glb I believe not it is accidental //
                \glft `I don't belive that it is accidental.' //
           \endgl\xe
%          \end{itemize}
        \end{block}
        \begin{block}{Annotation example 2: Lexical and function words}
            \ex\label{utenat}\begingl
                \gla Per har spist et eple og en pære //
                \glb Per has eaten an apple and a pear //
                \glft `Per has eaten an apple and a pear.' //
           \endgl\xe
        \end{block}

    \end{column}
    \begin{column}{0.5\textwidth}
        \begin{block}{Tittel}
            \begin{itemize}
                \item Foo
                \item Bar
                \item Blerg
            \end{itemize}
        \end{block}
    \end{column}
\end{columns}
\end{document}

